\documentclass{article}
\usepackage{tikzit}
\documentclass{article}
\usepackage{tikzit}
\documentclass{article}
\usepackage{tikzit}
\documentclass{article}
\usepackage{tikzit}
\input{sample.tikzstyles}


\begin{document}

This is a demonstration of \texttt{tikzit.sty}, which provides some convenience macros for including \texttt{.tikz} files generated by TikZiT. Note this file is optional, however if you choose to omit it from your \LaTeX{} source, you should at least declare the layers, dummy properties, and \texttt{none} style from \texttt{tikzit.sty} for TikZiT figures to build correctly.

A centered tikz picture:
\ctikzfig{fig}

A tikz picture as part of mathematics:
\begin{equation}
\tikzfig{fig} \ =\ 
\tikzfig{fig}
\end{equation}

It is also possible to paste a \texttt{tikzpicture} directly from TikZiT, without using the \texttt{$\backslash$tikzfig} macro. In that case, the \texttt{tikzfig} option should be given to the \texttt{tikzpicture} environment to get the same baseline and scaling as the other figures:
\[
\begin{tikzpicture}[tikzfig]
	\begin{pgfonlayer}{nodelayer}
		\node [style=red node] (0) at (0, 1) {};
		\node [style=blue node 2] (1) at (1, 0) {};
		\node [style=blue node] (2) at (-1, 0) {};
		\node [style=yellow square] (3) at (0, -1) {foo};
	\end{pgfonlayer}
	\begin{pgfonlayer}{edgelayer}
		\draw [in=-90, out=0] (3) to (1);
		\draw [bend right] (3) to (2);
		\draw (2) to (0);
		\draw (0) to (1);
	\end{pgfonlayer}
\end{tikzpicture}
\]


\end{document}




\begin{document}

This is a demonstration of \texttt{tikzit.sty}, which provides some convenience macros for including \texttt{.tikz} files generated by TikZiT. Note this file is optional, however if you choose to omit it from your \LaTeX{} source, you should at least declare the layers, dummy properties, and \texttt{none} style from \texttt{tikzit.sty} for TikZiT figures to build correctly.

A centered tikz picture:
\ctikzfig{fig}

A tikz picture as part of mathematics:
\begin{equation}
\tikzfig{fig} \ =\ 
\tikzfig{fig}
\end{equation}

It is also possible to paste a \texttt{tikzpicture} directly from TikZiT, without using the \texttt{$\backslash$tikzfig} macro. In that case, the \texttt{tikzfig} option should be given to the \texttt{tikzpicture} environment to get the same baseline and scaling as the other figures:
\[
\begin{tikzpicture}[tikzfig]
	\begin{pgfonlayer}{nodelayer}
		\node [style=red node] (0) at (0, 1) {};
		\node [style=blue node 2] (1) at (1, 0) {};
		\node [style=blue node] (2) at (-1, 0) {};
		\node [style=yellow square] (3) at (0, -1) {foo};
	\end{pgfonlayer}
	\begin{pgfonlayer}{edgelayer}
		\draw [in=-90, out=0] (3) to (1);
		\draw [bend right] (3) to (2);
		\draw (2) to (0);
		\draw (0) to (1);
	\end{pgfonlayer}
\end{tikzpicture}
\]


\end{document}




\begin{document}

This is a demonstration of \texttt{tikzit.sty}, which provides some convenience macros for including \texttt{.tikz} files generated by TikZiT. Note this file is optional, however if you choose to omit it from your \LaTeX{} source, you should at least declare the layers, dummy properties, and \texttt{none} style from \texttt{tikzit.sty} for TikZiT figures to build correctly.

A centered tikz picture:
\ctikzfig{fig}

A tikz picture as part of mathematics:
\begin{equation}
\tikzfig{fig} \ =\ 
\tikzfig{fig}
\end{equation}

It is also possible to paste a \texttt{tikzpicture} directly from TikZiT, without using the \texttt{$\backslash$tikzfig} macro. In that case, the \texttt{tikzfig} option should be given to the \texttt{tikzpicture} environment to get the same baseline and scaling as the other figures:
\[
\begin{tikzpicture}[tikzfig]
	\begin{pgfonlayer}{nodelayer}
		\node [style=red node] (0) at (0, 1) {};
		\node [style=blue node 2] (1) at (1, 0) {};
		\node [style=blue node] (2) at (-1, 0) {};
		\node [style=yellow square] (3) at (0, -1) {foo};
	\end{pgfonlayer}
	\begin{pgfonlayer}{edgelayer}
		\draw [in=-90, out=0] (3) to (1);
		\draw [bend right] (3) to (2);
		\draw (2) to (0);
		\draw (0) to (1);
	\end{pgfonlayer}
\end{tikzpicture}
\]


\end{document}




\begin{document}

This is a demonstration of \texttt{tikzit.sty}, which provides some convenience macros for including \texttt{.tikz} files generated by TikZiT. Note this file is optional, however if you choose to omit it from your \LaTeX{} source, you should at least declare the layers, dummy properties, and \texttt{none} style from \texttt{tikzit.sty} for TikZiT figures to build correctly.

A centered tikz picture:
\ctikzfig{fig}

A tikz picture as part of mathematics:
\begin{equation}
\tikzfig{fig} \ =\ 
\tikzfig{fig}
\end{equation}

It is also possible to paste a \texttt{tikzpicture} directly from TikZiT, without using the \texttt{$\backslash$tikzfig} macro. In that case, the \texttt{tikzfig} option should be given to the \texttt{tikzpicture} environment to get the same baseline and scaling as the other figures:
\[
\begin{tikzpicture}[tikzfig]
	\begin{pgfonlayer}{nodelayer}
		\node [style=red node] (0) at (0, 1) {};
		\node [style=blue node 2] (1) at (1, 0) {};
		\node [style=blue node] (2) at (-1, 0) {};
		\node [style=yellow square] (3) at (0, -1) {foo};
	\end{pgfonlayer}
	\begin{pgfonlayer}{edgelayer}
		\draw [in=-90, out=0] (3) to (1);
		\draw [bend right] (3) to (2);
		\draw (2) to (0);
		\draw (0) to (1);
	\end{pgfonlayer}
\end{tikzpicture}
\]


\end{document}

